\documentclass{report}
\usepackage[spanish]{babel}
\usepackage[utf8]{inputenc}
\usepackage{graphicx,wrapfig}
\begin{document}
	\tableofcontents
	\chapter{Carta de Presentación}
	Equipo de hyve code \\
	Presente\\
	\break
	Nos presentamos con el agrado de darles a conocer una propuesta para el modulo del proyecto actual conocido como "Bizion", el modulo a realizar sera el encargado
	de controlar la asignación de tareas, y lectura. \\
	
	Hyve code, una start-up derivada de lo que actualmente es cormago, integrado por un equipo multidisciplinario constituido por ingenieros en software, licenciados en informática, diseñadores gráficos, encargados de UX/UI y demás. Los cuales aportan sus ideas innovadoras y diferentes puntos de vista para el análisis de las situaciones.\\
	
	Se presenta con el primer proyecto de alto nivel donde el enfoque principal es el control de seguridad en plantas industriales, condómios y demás, aplicando en cualquier lugar donde se requiera seguridad.\\
	
	Dicho proyecto dirigido a largo plazo, permiten darle seguimiento al crecimiento del software, mantenimiento y control sobre el mismo, así mismo brindar servicio al cliente en caso de ser requerido como tal.\\
	
	Hacemos invitación a construir el panal tecnológico a nuestro lado.\\
	\begin{flushleft}
		\begin{wrapfigure}{r}{7 cm}
		\includegraphics[width=0.5\linewidth]{comargo}
		\includegraphics[width=0.5\linewidth]{hyvecode}	
		\end{wrapfigure}
	\end{flushleft}		
	
	
	
	\chapter{Cumplimientos de Elementos a Evaluar}
		\section{Certificaciones}
			\subsection{Certificaciones del Proveedor}
			Actualmente no se cuenta con certificaciones generales al equipo
			\subsection{Certificaciones de Perfiles}
			
		\section{Experiencia en Proyectos}
			El equipo cuenta con mas de 5 proyectos en portafolio sin contar los proyectos personales de cada integrante.
			
		\section{Niveles de Servicio}
			SLA de primera respuesta: 1 - 2 horas.\\
			SLA de resolución: 2-4 horas, según complejidad.
			
		\section{Cumplimiento de Políticas y/o Programas de Sustentabilidad}
			Actualmente no se cuenta con algún programa de inclusión para elementos con discapacidad.
		\section{Cumplimiento de Requisitos Formales}		
			Hyve code y Cormago S.A. de C.V. son accesibles con su información al momento de ser solicitada.
		\section{Garantía de Software}
			El software a desarrollar al tener un modelo de negocios basado en membresías, se estima que tendrá una garantía extensa ya que se tiene contemplado darle mantenimiento por al menos 5 años
		\section{Componente técnico}
			El control de versionamiento sera por medio de GitHub y varias branches de prueba.
		\section{Encargado Técnico}
			Actualmente en el proyecto, el elemento que hace la función de Product Owner es: Rutilo Cruz, líder del equipo de Hyve Code.
	
	\chapter{Propuesta Técnica}
		\section{Resumen Ejecutivo}
		De acuerdo a las indicaciones y necesidades establecidas en la reunión anterior con el equipo, se requiere el diseño y realización de un modulo para el software "Bizion", dicho modulo sera el encargado de permitir a administradores asignar tareas y recorridos a los integrantes de su plantilla de equipo. El objetivo principal de este modulo es ayudar a la organización y permitir que la comunicación entre puntos de vigilancia sea optima y efectiva. \\
		
		El software a desarrollar es un modulo de un proyecto en el cual están trabajando actualmente, en términos generales funcionara como un sistema de control para equipos de seguridad, así mismo brindara estadísticas de incidencias u anormalidades encontradas, contara con dos modalidades de uso, online y offline, permitiendo trabajar con la aplicación sin necesidad de tener una conexión constante, unicamente para hacer el envió de datos.
	
		
		\section{Etapas del Proyecto}
			\subsection{Planificación del Proyecto}
					se contempla la complejidad y duración de realización generando una grafica de Grantt 
					\newpage
		\section{Análisis de requerimiento}
				El análisis se llevo a cabo por medio de reuniones presenciales con los distintos actores y algunos usuarios preliminares del sistema, con el fin de generar un listado común de funcionalidades, las cuales el sistema debe cumplir para su optimo despliegue al publico.
			\subsection{Análisis y Diseño}
			\textbf{Diseño funcional}:se realizaran dos propuestas de diseño y maquetado para acercar el sistema los usuarios, así también se puede tener feedback sobre el mismo y los cambios pertinentes que se podrían agregar. \\
			\textbf{Plan de pruebas}: gracias a nuestro encargado de QA se realizaran diversas pruebas no solo de funcionalidad si no tambien del mismo sistema y servidores para comprobar su robustez ante el uso publico. \\
			\textbf{Casos de prueba}:se analizan las diferentes funcionalidades del sistema para generar casos de prueba mas cercanos a la realidad para estimar el uso y funcionalidad del sistema en entornos reales. 
			\subsection{Construcción}
			Durante esta etapa del proyecto se desarrollara no solo la parte visual en código si no también la parte funcional de este, permitiendo que tenga funcionalidad online tanto para web, como dispositivos android u IOS.

		\section{Metodología de Trabajo}
			En este proyecto se utiliza la metodología SCRUM , principalmente por el tamaño del equipo ademas de que la meta principal del mismo equipo es poder mantener el nivel de conocimiento acerca del proyecto como algo publico dentro del mismo, así no hay retrasos o perdidas de información acerca del mismo, el control usando esta metodología es atravez de JIRA un software de atlassian, en el cual se puede generar un control de tareas, epicas, errores, y asi mismo la organización del SPRINT semanal para el control de actividades. \\
			
			El proceso Scrum se inicia con un Sprint 0, previa al inicio de la primera iteración de desarrollo, en que se define la visión del proyecto y se elabora la lista de objetivos/requisitos priorizada del producto (Product Backlog), que actúa como plan del proyecto. De manera regular el Product Owner maximiza la utilidad de lo que se desarrolla y el Retorno de Inversión del proyecto mediante la re planificación de objetivos / requisitos en una reunión de gestión de cambios llamada Product Backlog Refinement.
			
			\subsection{Metodología Prototipo}
			La construcción general se llevará a cabo mediante una metodología de prototipos, que comprende un conjunto de fases, actividades y tareas destinadas a lograr el éxito en un proyecto. Cada actividad está respaldada por una amplia variedad de plantillas de documentos que permiten al Jefe de Proyecto reducir significativamente el tiempo necesario para completarlas, garantizando así su ejecución de manera clara y concisa. Además, se incluyen "procesos de gestión" que permiten a los administradores de proyectos supervisar y controlar la producción de manera efectiva.\\
			
			La adopción de esta metodología por parte de los Jefes de Proyecto aumenta considerablemente la probabilidad de éxito en sus proyectos, basándose en la experiencia acumulada a partir de numerosos proyectos y una diversidad de productos y servicios en una amplia gama de industrias.\\
			
			El propósito principal de esta metodología es ofrecer dos enfoques de gestión principales: la entrega de resultados tempranos y una planificación de proyecto ágil pero bajo control. Esta metodología consta de fases iniciales en las cuales se establecen los requisitos por parte del cliente y se lleva a cabo la construcción del producto. Cada una de estas fases posee su propia metodología específica.
			
				\subsubsection{Arquitectura y Usabilidad}
					Dispositivos: el foco principal es el uso en móviles por medio de una aplicación y el uso en web para el análisis y seguimiento de métricas por medio de interfaces, tablas o graficas según sea el caso. \\
					\break
					Procesos guiados: el foco principal para acercar al usuario con el sistema es el uso de Procesos guiados, este permitirá mostrar el uso completo del sistema sin necesidad de alguna capacitación humana.\\
					
					\subsubsection{Diseño}
						Se realizara un análisis basado en estudio de mercado y el user person para que sea el indicado a el publico dirigido.
						
					\subsubsection{Front End}	
					El desarrollo del Front-End se basa en la lógica prevista para asegurar una visualización adecuada en múltiples dispositivos, navegadores y sistemas operativos, tanto en entornos de escritorio como en dispositivos móviles, y en una variedad de tamaños de pantalla. Para lograr esto, se empleará React.js para la versión web y React Native para la versión móvil.\\
					\break
					En primer lugar, se analizarán las maquetas para identificar elementos recurrentes y comunes, a partir de los cuales se creará la base de estilos CSS para todas las plantillas. Como paso siguiente, se generarán archivos CSS específicos para cada plantilla, con el fin de distinguir los elementos particulares y únicos de cada estructura y simplificar su integración.\\
					\break
					Las estructuras de React.js, CSS y JS estarán diseñadas de acuerdo a los estándares necesarios para garantizar un funcionamiento correcto y una visualización óptima en todos los dispositivos, utilizando React Native para la versión móvil, lo que permitirá una experiencia fluida y nativa en dispositivos móviles.
		\section{Duración del Proyecto y Grantt Preliminar}	
			\subsection{Plazo}
				6 meses.
	
	\chapter{Propiedad Intelectual}
	
	\chapter{Presentación de la Empresa}
		\section{Identificación de la Empresa}
		\section{Representante Legal}
		\section{Descripción}
		\section{Nuestra Propuesta de Valor}
		\section{Proyectos y servicios recientes}
		\section{Clientes recientes}
	
\end{document}